%%%%%%%%%%%%%%%%%%%%%%%%%%%%%%%%%%%%%%%%%%%%%%%%%%%%%%%%%%%%%%%%%%%%%%
% BAB PENDAHULUAN:
%=====================================================================
\pagenumbering{arabic}
\renewcommand{\thechapter}{\Roman{chapter}}
\addtocontents{toc}{\vskip10pt}
\chapter{PENDAHULUAN}
\renewcommand{\thechapter}{\arabic{chapter}}
\pagestyle{konten}
%---------------------------------------------------------------------

%=====================================================================
\section{Latar Belakang}
%=====================================================================

\textit{Template} \LaTeX{} untuk Tugas Akhir ini terdiri dari beberapa bagian, di mana \textit{file} \texttt{main.tex} merupakan bagian utama yang berguna untuk menyatakan dan mengatur kerangka serta \textit{input} yang digunakan dalam dokumen Tugas Akhir. \textit{File} Tugas Akhir dalam format \texttt{$^*$.pdf} akan dapat dihasilkan dengan mengkompilasi \texttt{main.tex} menggunakan \textit{compiler} Lua\LaTeX. Selain itu, bagian lain yang juga penting dan dapat diubah adalah:

\setlist{nolistsep}
\begin{enumerate}[noitemsep]
    %
    \item \textit{File} \texttt{informasi.tex}, untuk memuat informasi seperti nama mahasiswa/(i), nomor induk mahasiswa, nama dosen pembimbing, NIP dosen pembimbing, nama dosen penguji, NIP dosen penguji, nama kepala departemen, NIP kepala departemen, nama departemen hingga nama perguruan tinggi, dan tanggal pengesahan Tugas Akhir.
    %
    \item \textit{Folder} \texttt{gambar}, berisi \textit{file} gambar dengan format \texttt{jpg}, \texttt{jpeg}, \texttt{png}, \texttt{pdf}, \texttt{tiff} dan/atau \texttt{eps} yang akan dimuat dalam dokumen Tugas Akhir.
    %
    \item \textit{Folder} \texttt{halaman-depan}, berisi \textit{file} \texttt{$^*$.tex} dan gambar yang akan dimuat di bagian depan, sebelum bab Pendahuluan, dokumen Tugas Akhir. Abstrak dalam bahasa Indonesia dituliskan dalam \textit{file} \texttt{abstrak.tex}, sementara abstrak dalam bahasa Inggris dituliskan dalam \textit{file} \texttt{abstract.tex}.
    %
    \item \textit{File} \texttt{kodeUnit.tex}, untuk memuat kode ringkas dari simbol, satuan, dan singkatan yang digunakan dalam dokumen Tugas Akhir.
    %
    \item \textit{Folder} \texttt{konten}, berisi \textit{file} \texttt{$^*$.tex} dari bagian-bagian yang akan dimasukkan ke dalam Tugas Akhir, dari bab Pendahuluan hingga Penutup.
    %
    \item \textit{File} \texttt{pustaka.bib}, merupakan \textit{file} BibTeX yang berisi daftar referensi yang digunakan dalam dokumen Tugas Akhir.
    %
    \item \textit{Folder} \texttt{halaman-belakang}, berisi \textit{file} \texttt{biografi.tex} untuk memuat biografi singkat mahasiswa/(i) dan \textit{folder} \texttt{lmapiran} yang berisi \textit{file} \texttt{$^*$.tex} untuk memuat lampiran.
\end{enumerate}


%=====================================================================
\section{Rumusan Masalah}
%=====================================================================

Format pengetikan pada \textit{template} \LaTeX{} Tugas Akhir ini telah menyesuaikan dengan ketentuan yang berlaku di Departemen Fisika, \its{} (ITS). Di antaranya, jenis dan ukuran kertas, jarak spasi, jarak tepi (\textit{margin}), dan jenis huruf. Sehingga, mahasiswa/(i) dapat fokus kepada isi dan substansi dari Tugas Akhir yang disusun.



%=====================================================================
\section{Tujuan}
%=====================================================================

%Lorem ipsum.
%\setlist{nolistsep}
%\begin{enumerate}[noitemsep]
%    \item Lorem ipsum.
%    \item ...
%\end{enumerate}


%===================================================================
\section{Batasan Masalah}
%===================================================================

%Lorem ipsum.
%\setlist{nolistsep}
%\begin{enumerate}[noitemsep]
%    \item Lorem ipsum.
%    \item ...
%\end{enumerate}


%=====================================================================
\section{Manfaat}
%=====================================================================

%Lorem ipsum.
%\setlist{nolistsep}
%\begin{enumerate}[noitemsep]
%    \item Lorem ipsum.
%    \item ...
%\end{enumerate}


%=====================================================================
\section{Sistematika Penulisan}
%=====================================================================

\vspace{3mm}

% CATATAN: Bila diperlukan, sistematika penulisan ini dapat dipisah menjadi
% dua bagian agar bisa terpisah menjadi dua halaman.

\begin{tabular}{ p{0.15\textwidth} p{0.05\textwidth} p{0.60\textwidth}}
    \textbf{BAB I}  & \textbf{:} &
    \textbf{PENDAHULUAN} \\
    && 
    Tuliskan paparan singkat mengenai isi bagian Pendahuluan di sini \\ 
    %
    \textbf{BAB II}  & \textbf{:} &
    \textbf{TINJAUAN PUSTAKA} \\
    &&
    Tuliskan paparan singkat mengenai isi bagian Tinjauan Pustaka di sini \\
    % 
    \textbf{BAB III}  & \textbf{:} &
    \textbf{METODOLOGI} \\
    && 
    Tuliskan paparan singkat mengenai isi bagian Metodologi di sini \\ 
    % 
    \textbf{BAB IV} & \textbf{:} &
    \textbf{ANALISIS DAN PEMBAHASAN} \\
    && 
    Tuliskan paparan singkat mengenai isi bagian Analisis dan Pembahasan di sini \\
    %
    \textbf{BAB V} & \textbf{:} &
    \textbf{PENUTUP} \\
    && 
    Tuliskan paparan singkat mengenai isi bagian Penutup di sini \\
    %
    \multicolumn{3}{l}{\textbf{LAMPIRAN}}\\
    &&
    Tuliskan paparan singkat mengenai isi bagian Lampiran di sini \\
    %
\end{tabular}

%%%%%%%%%%%%%%%%%%%%%%%%%%%%%%%%%%%%%%%%%%%%%%%%%%%%%%%%%%%%%%%%%%%%%%